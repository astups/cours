\section*{Considérations Générales}
\textbf{Veillez à vous protéger ! Ces règles ne sont pas là pour vous embêter mais pour vous protéger !}

Les quatre commandements du bricolage~:

\begin{minipage}{0.45\columnwidth}
\begin{center}
%\hspace{0.05\columnwidth} $\bullet$ 
$\bullet$ {\tailleRegles Protégez vous}

$\bullet$ {\tailleRegles Restez attentifs}
\end{center}
\end{minipage}
\begin{minipage}{0.53\columnwidth}
\begin{center}
$\bullet$ {\tailleRegles Entretenez vos outils}

$\bullet$ {\tailleRegles Pensez aux autres}
\end{center}
\end{minipage}

\vspace{0.05\columnwidth}

Veillez à ce que les règles de sécurité soit respectée. Voici des conseils de base pour éviter de vous blesser ou de blesser votre entourage.

\textbf{Protégez vous !} Lorsque vous utilisez un outil pensez à vous équiper de protections de base: gants, lunettes de protections, etc.

\textbf{Avant d'utiliser un outil}, vérifiez que vous avez pris toutes les précautions, referez vous à la fiche bricolage. En cas de doute, demandez de l'aide à quelqu'un.

\textbf{Nettoyez votre plan de travail !} Tout ce qui traîne peut vous gêner, se casser, voire même vous blesser !

\textbf{Utilisez un outil adapté}, pour votre bricolage ! Un couteau n'est pas un tournevis !

\textbf{Faites attention à la trajectoire de l'outil !} Retroussez vos manches, attachez vous les cheveux, enlevez vos bijoux ! Regardez aussi ou vous mettez vos mains !

\textbf{Pensez aux autres !} Vérifiez que les personnes qui vous entourent ne vous gênerons pas dans votre travail, vous risquez de vous blesser et de les blesser !

\textbf{Vérifiez votre matériel !} Un équipement en bon état et entretenu garantis plus de sécurité !

\textbf{Attention aux câbles !} Ils sont très rapidement source de dangers, un outil qui bouge/tombe, un câble coupé/fondu sont très fréquents en bricolage !
