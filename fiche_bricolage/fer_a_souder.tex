\section*{Fer à souder}
Un fer à souder est un outil permettant de souder deux composants entre eux en faisant fondre un métal d’apport (fil de plomb et d’étain). Il est constitué d’un manche en plastique, d’une partie métallique chauffante et d’une panne.

\textbf{Un fer à souder est par définition très chaud~:} Éloignez tout matériau inflammable du plan de travail (attention aux vêtements et aux cheveux). Il faut toujours le manipuler par le manche en plastique, et ne jamais toucher la partie métallique. Le fer doit toujours être reposé sur un support stable prévu à cet effet entre chaque utilisation. Ne pas utiliser de gants jetables en latex pour se protéger les mains, car ceux-ci peuvent fondre sous la chaleur, et devenir très dangereux.

\textbf{Le métal d’apport est constitué de produits toxiques (plomb, étain)~:} Il est donc important de ne pas respirer les vapeurs issues de la soudure et de travailler dans un local aéré. Il faut toujours se laver les mains une fois la soudure finie, ne jamais mettre le fil de soudure à la bouche, et ne pas souder à proximité de nourriture.

\todo{Insérer image refaite d'n fer à souder}

\begin{itemize}
\item Toujours nettoyer son plan de travail avant et après la soudure.
\item L’embout du fer à souder doit être nettoyé régulièrement à l’aide de l’éponge prévue à cet effet (et seulement à cet effet).
\item Ne jamais souder sur un appareil sous tension. Vérifiez toujours que l’appareil est bien débranché et qu’aucune pile n’est présente.
\item Enfin, après utilisation, veilliez à ce que le fer soit bien éteint et correctement rangé.
\end{itemize}
