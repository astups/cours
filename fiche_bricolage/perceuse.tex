\section*{Perceuse et Visseuse}
Ce sont des outils très largement présents dans les foyers, cependant il est utile de préciser quelques informations concernant leurs utilisations, pour votre bien et celui du matériel.

\subsection*{Générale}
\begin{itemize}
\item Utilisez le \textbf{foret ou l'embout adapté} à votre usage, cf. partie tournevis pour le choix des embouts pour visser, si vous avez un doute pour la mèche demandez conseil.

\item Faire un \textbf{pré-trou facilite le perçage}, par exemple dans du bois ou bien si vous voulez percer des trous de diamètre important.

\item \textbf{Contrôlez la vitesse} de l'outil, en effet plus la surface est dur plus le couple doit être élevé : sur la machine on peut régler autour du mandrin la force de la machine, un nombre généralement de 1 à 9 et position perceuse.

\item Bien \textbf{maintenir en place} ce que vous percez (ou vissez) avec, par exemple, un étaux. Cela évite que l'objet ne soit entraîné par la machine ce qui est dangereux.
\end{itemize}

\subsection*{Cas particulier}
\begin{itemize}
\item Utilisez \textbf{des gants ou des lunettes} quand il y a des risques de brûlures ou de projections, par exemple percer du bois peut faire des projections dans les yeux.

\item Percez du métal en n'utilisant que des \textbf{mèches de type HSS et huilez} la surface.
\end{itemize}
