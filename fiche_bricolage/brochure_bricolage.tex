\documentclass[foldmark,notumble]{leaflet}
%\documentclass[foldmark,tumble]{leaflet} %Peut être nécessaire pour l'impression

\usepackage[T1]{fontenc}
\usepackage[utf8]{inputenc}
\usepackage{url}
\usepackage{graphicx}
\usepackage[francais]{babel}

\AtBeginDocument{\def\labelitemi{$\bullet$}}
\AtBeginDocument{\def\labelitemii{$\circ$}}
\AtBeginDocument{\def\labelitemiii{--}}
%\AtBeginDocument{\def\labelitemiv{$\bullet$}}

\title{Conseils de Bricolage}
\author{ASTUPS}
\date{2014/2015 - v1.0}

%Commande du paquet leaflet
%CutLine : dessine une ligne verticale en pointillés avec un ciseau, en paramètre la page concernée
%CutLine* : idem à CutLine sans le ciseau
%AddToBackground : Met une image en font d'une portion de la brochure
%AddToBackground* : Met une image en font d'une page entière (trois portions de brochure)

%\AddToBackground{6}{%  Background of a small page
%  \put(0,0){\textcolor{YellowOrange}{\rule{\paperwidth}{\paperheight}}}}


%\AddToBackground*{2}{% Background of a large page
%  \put(\LenToUnit{.5\paperwidth},\LenToUnit{.5\paperheight}){%
%    \makebox(0,0)[c]{%
%      \resizebox{.9\paperwidth}{!}{\rotatebox{35.26}{%
%        \textsf{\textbf{\textcolor{LIGHTGRAY}{CLEMSON}}}}}}}}

\CutLine*{1}
\CutLine*{3}
\CutLine*{4}
\CutLine*{6}

\begin{document}
\thispagestyle{empty}

\maketitle

\section{Considérations Générales}
Lorem ipsum dolor sit amet, consectetur adipiscing elit. Nulla enim quam, porttitor sed massa vel, aliquet sagittis tortor. Suspendisse est eros, faucibus non vehicula eget, rutrum at purus. Aliquam auctor vehicula purus, consequat luctus lectus iaculis ultrices. Aliquam tempor dictum ullamcorper. Ut a facilisis lorem, vel tempus eros. Sed elit odio, feugiat vel molestie id, congue in nisl. Aliquam id erat non mi dapibus vulputate. Suspendisse potenti. Aenean tincidunt porta dapibus. Donec euismod eleifend urna, nec pharetra ex iaculis quis.


\newpage
\section{Tournevis et Clés}
Lorem ipsum dolor sit amet, consectetur adipiscing elit. Nulla enim quam, porttitor sed massa vel, aliquet sagittis tortor. Suspendisse est eros, faucibus non vehicula eget, rutrum at purus. Aliquam auctor vehicula purus, consequat luctus lectus iaculis ultrices. Aliquam tempor dictum ullamcorper. Ut a facilisis lorem, vel tempus eros. Sed elit odio, feugiat vel molestie id, congue in nisl. Aliquam id erat non mi dapibus vulputate. Suspendisse potenti. Aenean tincidunt porta dapibus. Donec euismod eleifend urna, nec pharetra ex iaculis squids.

Lorem ipsum dolor sit amet, consectetur adipiscing elit. Sed malesuada laoreet libero, vitae eleifend metus ultricies eget. Quisque ultrices turpis et orci congue dapibus volutpat sed massa. In lacinia nulla vitae augue imperdiet porttitor. Fusce porta mauris sit amet purus elementum, eget rhoncus lacus tincidunt. Fusce a tempor elit, nec varius erat. Fusce nec vestibulum erat. Vestibulum in lobortis mauris. Phasellus eu volutpat magna, et semper odio. Interdum et malesuada fames ac ante ipsum primis in faucibus. Praesent enim sem, hendrerit eget dictum vel, vehicula vel felis. Suspendisse ultricies eget ex et ullamcorper. Phasellus vitae tellus tincidunt augue gravida ultricies. Mauris gravida lectus vel velit varius, vitae vulputate justo aliquam.


\section{Perçeuse et Viseuse}
Lorem ipsum dolor sit amet, consectetur adipiscing elit. Nulla enim quam, porttitor sed massa vel, aliquet sagittis tortor. Suspendisse est eros, faucibus non vehicula eget, rutrum at purus. Aliquam auctor vehicula purus, consequat luctus lectus iaculis ultrices. Aliquam tempor dictum ullamcorper. Ut a facilisis lorem, vel tempus eros. Sed elit odio, feugiat vel molestie id, congue in nisl. Aliquam id erat non mi dapibus vulputate. Suspendisse potenti. Aenean tincidunt porta dapibus. Donec euismod eleifend urna, nec pharetra ex iaculis quis.

Lorem ipsum dolor sit amet, consectetur adipiscing elit. Sed malesuada laoreet libero, vitae eleifend metus ultricies eget. Quisque ultrices turpis et orci congue dapibus volutpat sed massa. In lacinia nulla vitae augue imperdiet porttitor. Fusce porta mauris sit amet purus elementum, eget rhoncus lacus tincidunt. Fusce a tempor elit, nec varius erat. Fusce nec vestibulum erat. Vestibulum in lobortis mauris. Phasellus eu volutpat magna, et semper odio. Interdum et malesuada fames ac ante ipsum primis in faucibus. Praesent enim sem, hendrerit eget dictum vel, vehicula vel felis. Suspendisse ultricies eget ex et ullamcorper. Phasellus vitae tellus tincidunt augue gravida ultricies. Mauris gravida lectus vel velit varius, vitae vulputate justo aliquam.


\section{Fer à souder}
Lorem ipsum dolor sit amet, consectetur adipiscing elit. Nulla enim quam, porttitor sed massa vel, aliquet sagittis tortor. Suspendisse est eros, faucibus non vehicula eget, rutrum at purus. Aliquam auctor vehicula purus, consequat luctus lectus iaculis ultrices. Aliquam tempor dictum ullamcorper. Ut a facilisis lorem, vel tempus eros. Sed elit odio, feugiat vel molestie id, congue in nisl. Aliquam id erat non mi dapibus vulputate. Suspendisse potenti. Aenean tincidunt porta dapibus. Donec euismod eleifend urna, nec pharetra ex iaculis quis.

Lorem ipsum dolor sit amet, consectetur adipiscing elit. Sed malesuada laoreet libero, vitae eleifend metus ultricies eget. Quisque ultrices turpis et orci congue dapibus volutpat sed massa. In lacinia nulla vitae augue imperdiet porttitor. Fusce porta mauris sit amet purus elementum, eget rhoncus lacus tincidunt. Fusce a tempor elit, nec varius erat. Fusce nec vestibulum erat. Vestibulum in lobortis mauris. Phasellus eu volutpat magna, et semper odio. Interdum et malesuada fames ac ante ipsum primis in faucibus. Praesent enim sem, hendrerit eget dictum vel, vehicula vel felis. Suspendisse ultricies eget ex et ullamcorper. Phasellus vitae tellus tincidunt augue gravida ultricies. Mauris gravida lectus vel velit varius, vitae vulputate justo aliquam.


\section{Colles et Pistolet à colle}
Lorem ipsum dolor sit amet, consectetur adipiscing elit. Nulla enim quam, porttitor sed massa vel, aliquet sagittis tortor. Suspendisse est eros, faucibus non vehicula eget, rutrum at purus. Aliquam auctor vehicula purus, consequat luctus lectus iaculis ultrices. Aliquam tempor dictum ullamcorper. Ut a facilisis lorem, vel tempus eros. Sed elit odio, feugiat vel molestie id, congue in nisl. Aliquam id erat non mi dapibus vulputate. Suspendisse potenti. Aenean tincidunt porta dapibus. Donec euismod eleifend urna, nec pharetra ex iaculis quis.

Lorem ipsum dolor sit amet, consectetur adipiscing elit. Sed malesuada laoreet libero, vitae eleifend metus ultricies eget. Quisque ultrices turpis et orci congue dapibus volutpat sed massa. In lacinia nulla vitae augue imperdiet porttitor. Fusce porta mauris sit amet purus elementum, eget rhoncus lacus tincidunt. Fusce a tempor elit, nec varius erat. Fusce nec vestibulum erat. Vestibulum in lobortis mauris. Phasellus eu volutpat magna, et semper odio. Interdum et malesuada fames ac ante ipsum primis in faucibus. Praesent enim sem, hendrerit eget dictum vel, vehicula vel felis. Suspendisse ultricies eget ex et ullamcorper. Phasellus vitae tellus tincidunt augue gravida ultricies. Mauris gravida lectus vel velit varius, vitae vulputate justo aliquam. 


\section*{Scie sauteuse} % [et disqueuse]}
Une scie sauteuse est un outil de découpe qui utilise une lame entraînée de bas en haut pour couper (d'où le nom de "sauteuse"). Il faut donc faire bien attention à ce mouvement de va et vient de cette lame pour ne pas se blesser ou abîmer le matériel proche de la zone de coupe.

Choisir une lame adaptée au matériau à couper~:
\begin{itemize}
\item petites dents pour les métaux, \textbf{Utiliser de l'huile et lames de type HSS.}
\item dents larges et alternées pour les découpes grossières dans du bois,
\item dents petites et alternées pour les découpes précises dans du bois,
\item \textbf{souvent marqué sur la lame}.
\end{itemize}

Choisir sa vitesse~:
\begin{itemize}
\item vitesse réduite pour les plastiques et métaux,
\item vitesse élevée pour le bois.
\end{itemize}

Attention à bien garder une \textbf{orientation constante} de la lame ; l'appareil peut tourner mais ne pas le pencher dans un sens puis l'autre.

Après une coupe les bords peuvent être coupants (surtout pour les métaux), pensez donc à «ébarber» à la lime ou à l'abrasif (ex: papier de verre).


\newpage
\section{Contact}

\begin{tabular}{ll}
Mail~: & \url{astups@gmail.com} \\
Forum~: & \url{http://astups.forumactif.com} \\
GitHub~: & \url{https://github.com/astups} \\
\end{tabular}

\end{document}
