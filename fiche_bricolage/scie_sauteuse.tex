\section*{Scie sauteuse} % [et disqueuse]}
Une scie sauteuse est un outil de découpe qui utilise une lame entraînée de bas en haut pour couper (d'où le nom de "sauteuse"). Il faut donc faire bien attention à ce mouvement de va et vient de cette lame pour ne pas se blesser ou abîmer le matériel proche de la zone de coupe.

Choisir une lame adaptée au matériau à couper~:
\begin{itemize}
\item petites dents pour les métaux, \textbf{Utiliser de l'huile et lames de type HSS.}
\item dents larges et alternées pour les découpes grossières dans du bois,
\item dents petites et alternées pour les découpes précises dans du bois,
\item \textbf{souvent marqué sur la lame}.
\end{itemize}

Choisir sa vitesse~:
\begin{itemize}
\item vitesse réduite pour les plastiques et métaux,
\item vitesse élevée pour le bois.
\end{itemize}

Attention à bien garder une \textbf{orientation constante} de la lame ; l'appareil peut tourner mais ne pas le pencher dans un sens puis l'autre.

Après une coupe les bords peuvent être coupants (surtout pour les métaux), pensez donc à «ébarber» à la lime ou à l'abrasif (ex: papier de verre).
